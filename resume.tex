\documentclass[letterpaper,12pt]{resume}
\pagestyle{plain}
\begin{document}

\let\olditemize\itemize
\renewcommand{\itemize}{
  \olditemize
  \setlength{\itemsep}{1pt}
  \setlength{\parskip}{0pt}
  \setlength{\parsep}{0pt}
}

\author{James Brisson}
\email{theotherjimmy@gmail.com}
\phone{(512) 497 9486}
\streetaddress{912 \#B Romeria Drive}
\citystatezip{Austin, Tx 78757}
\maketitle

\section{Skills}
\noindent
\begin{tabular}{p{0.4\textwidth}p{0.55\textwidth}}
  Programming Languages&
  Rust, C[++], Haskell, \LaTeX\, Python, Bash, Make \\
  \noalign{\smallskip}
  Assemblies&
  ARMv7E-M (GAS), Intel i686 (GAS), Freescale 6812, TI TMS320C6000 DSP, \\
  \noalign{\smallskip}
  Software Development&
  Git, Merge Request workflow, GDB\\
\end{tabular}

\section{Education}
\noindent
\begin{tabular}{p{0.4\textwidth}p{0.55\textwidth}}
  BS in Electrical Engineering&
  UT Austin in December 2013\\
  \noalign{\smallskip}
  Tech Areas&
  Computer Design, Communications/Digital Signal Processing\\
  \noalign{\smallskip}
  Notable Classes&
  Operating Systems Honors (using C), Real-time DSP Lab, Computer Architecture, Real-time Embedded Systems\\
\end{tabular}

\subsection{Operating Systems}
\begin{itemize}
  \item
    Modified the Linux kernel scheduler and implemented several kernel modules
  \item
    Developed an exokernel for i686 in C and asm; ext2 drivers, self-hosting, graphical
\end{itemize}

\section{Professional Experience}
\subsection{May 2019 - Nov 2024: Embedded Firmware Engineer at Arm}
\begin{itemize}
  \item
    Benchmarked and improved TF-A memory footprint with custom tooling
  \item
    Snapshot debugger for Zepyhr with support for armv8m and TF-M
  \item
    Runtime debugger for Arm FVP including bridging to GDB and event logging
  \item
    Transitioned a Yocto Linux distribution from sysVinit to systemd
  \item
    Bench-marked and tweaked Kubernets for memory performance on an arm Linux single board computer
\end{itemize}
\subsection{May 2016 - April 2019: Mbed OS Software Engineer at Arm}
\begin{itemize}
  \item
    Extended build framework with new features while maintaining backwards compatibility
  \item
    Maintained an offline and online, multi-tenant IDE and testing infrastructure in Python
  \item
    Mentored open-source contributors to improve contribution quality and git history quality 
  \item
    Developed testing infrastructure that ran about 30k tests for each pull request
  \item
    Automated parts of the Pull Request(PR) process and developed a contribution model to handle 100 PRs a week
\end{itemize}

\subsection{March 2014 - May 2016: Research Assistant}
Working on several academic research projects, including buddythreads and
bubbles, mentioned below.

\subsection{Jan 2015 - May 2015 and Aug 2015 - Dec 2015: Teaching Assistant}
During Spring 2015 I was a TA for EE319K, Intro to Embedded Systems, and During
the Fall of the same year I was a TA for EE379K, Operating Systems.

\subsection{May 2013 - December 2013: Intern Silicon Labs}
\begin{itemize}
  \item
    Automated build system creating patch-able 8051 ROM and automated patch making
  \item
    Created testing framework for pre and post silicon (simulation, FGPA emulation, evaluation)
  \item
    Wrote firmware RC oscillator calibration algorithm and several patches
  \item
    Developed waveform capture tool for firmware symbols on a simulated 8051 processor
\end{itemize}

\section{Projects}
\begin{itemize} 
  \item 
    Cmsis-pack-manager: A highly concurrent download utility for CMSIS Packs written in Rust and Python
  \item
    Moses: A bluetooth controlled holonomic robot and controller
  \item
    Automated framework for estimating side channel capacities of contention based channels
  \item
    RASLib: intro to robotics library targeted at the TI Stellaris/Tiva Launchpads
  \item
    Custom Keyboard, Dactyl, with custom layout and firmware in Mbed OS
  \item
    Intelligent ground vehicle software design
\end{itemize}

\section{Research}
\subsection{Leg: An arm Emulator for fuzzing}
A prototype emulator foccussing on the bare minimum of hardware emulation to
boot TF-A, the ability to snapshot and restore quickly and multi-threading.
Leg emulates with both a simple interpreter and a JIT, with threads that fail
to take the JIT lock falling back to the interpreter.

\subsection{Buddythreads: Scheduler-Base Side Channel Defenses}
A modification to the Linux kernel that allows a process to request that it
should always be scheduled simultaneously with another, ``buddy'' thread.
This allows the buddy thread to make noise on shared resources that may be used
for side channel attacks.
I also developed and evaluated several methods for creating noise on these
shared resources and showed that perfect information of the victim's leakage is
sufficient to thwart attacks.

Submitted to ISCA 2016.

\end{document}
