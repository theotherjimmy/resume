\documentclass[letterpaper,12pt]{resume}
\pagestyle{plain}
\begin{document}

\let\olditemize\itemize
\renewcommand{\itemize}{
  \olditemize
  \setlength{\itemsep}{1pt}
  \setlength{\parskip}{0pt}
  \setlength{\parsep}{0pt}
}

\author{James Brisson}
\email{theotherjimmy@gmail.com}
\phone{(512) 497 9486}
\streetaddress{116 East 31\textsuperscript{st} Street Apt. 107}
\citystatezip{Austin, Tx 78705}
\maketitle

\section{Profile}
%\noindent
%I am a UT-trained Electrical Engineer, and am widely considered a Jedi-Master programmer by my peers.
%I have designed the entire stack for computers from transistors and computer architecture to operating systems and application level software. 
%I have performed significant work in all development phases of a project life cycle.
%In addition to computers, I also have experience with interfacing to reality via DSP.
%I am clearly passionate about Electrical Engineering as many of my hobbies are related to programming, embedded systems, and automation.

\section{Education}
\noindent
\begin{tabular}{p{0.4\textwidth}p{0.55\textwidth}}
  BS in Electrical Engineering&
  UT Austin in December 2013\\
  \noalign{\smallskip}
  Current Enrolment&
  Student at UT Austin\\
  \noalign{\smallskip}
  Tech Areas&
  Computer Design, Communications/Digital Signal Processing\\
  \noalign{\smallskip}
  Notable Classes&
  Operating Systems Honours (using C), Real-time DSP Lab, Computer Architecture, Real-time Embedded Systems\\
\end{tabular}

\section{Skills}
\noindent
\begin{tabular}{p{0.4\textwidth}p{0.55\textwidth}}
  Test and Measurement&
  Signal Generators, Oscilloscopes, Logic Analyser, Protocol Sniffer\\
  \noalign{\smallskip}
  Assemblies&
  Freescale 6812, LC3, TI TMS320C6000 DSP, Intel i686 (GAS), ARMv7E-M (GAS)\\
  \noalign{\smallskip}
  Mechanical CAD&
  AutoDesk Inventor, OpenSCAD, ImplicitCAD\\
  \noalign{\smallskip}
  Hardware Description Languages&
  Verilog, VHDL, C$\lambda$ash\\
  \noalign{\smallskip}
  Programming Languages&
  C[++], Haskell, \LaTeX\, Python, Common Lisp, Scheme, Clojure, Ruby, Perl, Matlab/Octave, Bash/Zsh, TCL, elisp, Make \\
  \noalign{\smallskip}
  Software Development&
  Emacs, Vim, Make, Ant, Eclipse, Xilinx ISE, Cadence, SimVision\\
\end{tabular}

\subsection{Operating Systems}
\begin{itemize}
  \item
    Modified the Linux kernel scheduler and implemented several kernel modules,
    both for research
  \item
    Developed an exokernel for the i686 in C and assembly; ext2 drivers, self-hosting, graphical
  \item
    Implemented an RTOS for the ARM Cortex-M in C and assembly
  \item
    Compiled custom kernels, Linux and Android, with patching
\end{itemize}

\pagebreak
\section{Research}
\subsection{Buddythreads: Scheduler-Base Side Channel Defenses}
A modification to the Linux kernel that allows a process to request that it
should always be scheduled simultaneously with another, ``buddy'' thread.
This allows the buddy thread to make noise on shared resources that may be used
for side channels.
I also developed and evaluated several methods for creating noise on these
shared resources and showed that perfect information of the victim's leakage is
sufficient to thwart attacks..

Submitted to ISCA 2016.

\subsection{Bubbles Secure Sharing System}
\noindent
A prototype security system that refactors the sharing out of mobile and web applications and makes it secure.
Sharing is then done on a 'Bubble' level, where information is grouped by the
user into a single package, or folder,  that may be shared.
Design principles of the system include:
\begin{itemize}
  \item
    Principal of Least surprise: minimise changes to the programming environment of the developer
  \item
    Interface integration between the application running in the Bubbles system and the system itself
  \item
    Lightweight: The containers that provide the security for the Bubbles system
    should have minimal performance impact
\end{itemize}
submitted to Okaland 2016

\section{Professional Experience}
\subsection{March 2014 - Ongoing: Staff Scientist at UT}
Working on several academic research projects, including buddythreads and
bubbles, mentioned above.

\subsection{Jan 2015 - May 2015 and Aug 2015 - Dec 2015: Teaching Assistant}
During Spring 2015 I was a TA for EE319K, Intro to Embedded Systems, and During
the Fall of the same year I was a TA for EE379K, Operating Systems.

\subsection{May 2013 - December2013: Intern Silicon Labs}
\begin{itemize}
  \item
    Automated build system creating patch-able 8051 ROM and automated patch making
  \item
    Created testing framework for pre and post silicon (simulation, FGPA emulation, evaluation)
  \item
    Wrote firmware RC oscillator calibration algorithm and several patches
  \item
    Developed waveform capture tool for firmware symbols on a simulated 8051 processor
\end{itemize}

\subsection{Summer 2010: Outback Director BTSR}
\begin{itemize}
  \item
    Managed 3 staffers leading a trek a week
  \item
    High adventure backpacking program
  \item
    Planned and tracked food and gear distribution across many campsites
\end{itemize}

\subsection{Summer 2009: Scout Skills Director BTSR}
\begin{itemize}
  \item
    Managed 3 staffers teaching classes
  \item
    Taught camping and outdoor skills
  \item
    Responsible for the teaching of 14 classes
  \item
    Lead toten chit and fireman chit sessions
\end{itemize}

\section{Community Service}
\begin{itemize}
  \item
    Mentor for 2013 and 2014 UT RAS Robotathon, Region V, and Mercury teams
  \item
    Eagle Scout Project --- build privacy fence for Humane Society of Williomson county
  \item
    Over 125 hours of community service through Boy Scout Troop 513
\end{itemize}

\section{Society Memberships}
\begin{itemize}
  \item
    Eagle Scout
  \item
    IEEE Robotics and Automation Society UT student branch
  \item
    IEEE UT student branch
\end{itemize}

\section{Projects}
\begin{itemize} 
  \item
    Aura: An Arch Linux package manager wrapper written in Haskell
  \item
    Moses: A bluetooth controlled holonomic robot
  \item
    Automated framework for estimating channel capacities of contention based channels
  \item
    RASLib: intro to robotics library targeted at the TI Stellaris/Tiva Launchpads
  \item
    Custom Keyboard, with custom layout and firmware
  \item
    Intelligent ground vehicle software design
  \item
    Remote controlled mobile couch with turn signals
  \item
    Robotics Booster-pack for TI Stellaris/Tiva Launchpads (PCB design)
  \item 
    Discussion Day: Tracking of student understanding through random sampling. Android application in Scheme
  \item
    Planar image stitching algorithm using phase correlation
  \item
    QPSK transceiver
\end{itemize}

\par
\begin{center}
  Recommendations available upon request
\end{center}

\end{document}
